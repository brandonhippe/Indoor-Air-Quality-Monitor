Our team wanted to make it easy to continuously monitor an indoor environment, and report data back to a host. Our aim was to make monitoring easy by making cheap, reliable devices that do not require frequent recharging. Three features of indoor air quality are expected to be monitored: 2.5 µm particle count or PM2.5, carbon dioxide (CO$_2$) concentration, and local air flow rate.\\
\\
Regarding CO$_2$, statistically significant decrements occurred in cognitive performance
(decision making, problem resolution) starting at 1000 ppm.  CO$_2$  concentration is also a
good proxy for ventilation; high CO$_2$ levels mean the room is poorly ventilated, which increases
the risk of passing airborne diseases such as COVID19.\\
\\
Regarding PM2.5, the WHO recommends an upper limit of 5 µg/m3 (microgram per cubic
meter) average annually and 15 µg/m3 average over a 24 hour period . \\
\\
An air speed sensor, or anemometer, can help us calculate how much air is flowing into or out of
a room and help understand why CO$_2$ and/or PM2.5 is high. \\
\\
The system is based on components past capstone teams have successfully incorporated
such as the TI MSP430 (selected for its particularly low-power sleep modes). In this iteration, a goal was to
replace the closed-source proprietary SmartMesh IP wireless system with the OpenWSN open-source wireless networking system.\\
\\
