Air quality is very important for safe working conditions. In every indoor space there should be a monitoring system for environmental risks and air pollutants such as CO$_2$ and fine particulate matter (PM2.5), as well as ventilation rates. Our team’s goal was to build a device to monitor elevated or dangerous quantities of these pollutants using as many commercial off-shelf components and open-source software as possible. Our aim was to create 3 to 10 wireless, battery powered initial prototypes that each have at least one year life span.\\
\\
This project is sponsored by the Wireless Environmental Sensor Technologies (WEST) Lab in the Electrical and Computer Engineering Department at Portland State, run by Dr. David C. Burnett. The project’s faculty advisor is Dr. John M. Acken.\\
\\
This project will be improving on the project from spring of 2022\cite{leon2023networked} to report readings of carbon dioxide, particulate matter, and airflow in an indoor environemnt instead of N$_2$O, temperature, and humidity in an outdoor environment.\\
\\
The team was successful in the project. We completed the project with 4 working nodes that communicate with each other using SmartMesh IP. The units can achieve a battery life of a year with reasonable capture times. Each unit is also equipped with the SGP30 CO$_2$ sensor, SPS30 PM2.5 sensor, and ClimateGuard Hotwire Anemometer to collect data. The system is housed in a cut acrylic box that can be mounted on walls or ceilings.\\
\\
This report details the background of this project, project requirements, and our approach to tackling this project. We also included appendices that contain the operation manual, issues we faced and solutions, and the components and software we used throughout the project.
