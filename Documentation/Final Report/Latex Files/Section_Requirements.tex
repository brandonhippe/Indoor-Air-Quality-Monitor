The sensor system must:
\begin{itemize}[label=$\bullet$]
  \item Sense PM2.5 and CO$_2$ often enough and accurately enough to ascertain indoor air quality relevant to occupants
  \item Include at least one node capable of sensing airspeed
  \item Maximize its battery life for sustained operation
  \item Locally store its measurement data
  \item Use as many commercial off-the-shelf components as possible
  \item Have an enclosure
  \item Wirelessly share its measurement data with a central monitoring system with communication range of at least 10 meters
  \item Utilize SmartMesh IP
  \item Have 3 iterations that cost no more than \$1,000 total
\end{itemize}

The sensor system should:
\begin{itemize}[label=$\bullet$]
  \item Include capability for any node to sense airspeed
  \item Have a battery life of at least 1 year
  \item Be open-source to the extent possible when using commercial off-the-shelf components
  \item Upgrade SmartMesh IP to utilize a low-power Wireless Sensor Network (WSN) system like OpenWSN 
  \item Utilize Texas Instruments MSP430/432 class microcontroller unit
  \item Have 10 iterations that cost no more than \$3,000 total
  \item Utilize 18650 lithium-ion battery cell(s)
\end{itemize}

The sensor system may:
\begin{itemize}[label=$\bullet$]
  \item Source all power from a single 18650 battery
  \item Monitor other environmental conditions such as temperature and humidity
  \item Include self-configuration capability to detect which sensors are connected and adjust sampling rates accordingly
  \item Be usable outdoors
  \item Include a visualization dashboard to monitor the data graphically
  \item Be able to incorporate many more (greater than 10) sensor modules
  \item Match the lifetime of a fire detector (approximately 10 years)
\end{itemize}